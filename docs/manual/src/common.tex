\begin{titlepage}
  \begin{center}

  {\Huge AXIS\_1553}

  \vspace{25mm}

  \includegraphics[width=0.90\textwidth,height=\textheight,keepaspectratio]{img/AFRL.png}

  \vspace{25mm}

  \today

  \vspace{15mm}

  {\Large Jay Convertino}

  \end{center}
\end{titlepage}

\tableofcontents

\newpage

\section{Usage}

\subsection{Introduction}

\par
AXIS 1553 is a transmit and receive for the MIL\-STD\-1553 bus. This core can run at full duplex even though
MIL\-STD\-1553 is half duplex. This core provides a simple axis streaming interace that uses tuser to extend the
bus to allow for command and data syncs to be choosen, along with setting or indicating more than a 4us delay. 
There are additional signals for frame errors, sync only, and parity errors. These can be used to manage issues
that present themeselves to the core. Internally this core generates its own enables to cycle data out at its 
synthetic sample rate. Data is transmitted based on a 1 MHz clock, but is sythetically generated to 2 Mhz sample rate.

\subsection{Dependencies}

\par
The following are the dependencies of the cores.

\begin{itemize}
  \item fusesoc 2.X
  \item iverilog (simulation)
  \item cocotb (simulation)
\end{itemize}

\input{src/fusesoc/depend_fusesoc_info.tex}

\subsection{In a Project}
\par
Connect the device to your AXIS bus. TUSER is used to set or check various data status such as command/data packet mode.

\par
TDATA input should contain the 16 bit data payload. TUSER is a 4 bit command register
that takes a description what type of data it is (command or data) and other options
described below.

TUSER = {D,TYY} (3 downto 0)
\begin{itemize}
\item TYY = TYPE OF DATA
  \begin{itemize}
    \item 000 N/A (IF DATA VALID, SYNC IS NOT VALID AND DATA IS NOT AS WELL)
    \item 001 REG (NOT IMPLEMENTED)
    \item 010 DATA
    \item 100 CMD/STATUS
  \end{itemize}
  \item D = DELAY ENABLED
  \begin{itemize}
    \item 1 = 4 us delay enabled.
    \item 0 = no delay between transmissions.
  \end{itemize}
\end{itemize}

\section{Architecture}
\par
This core is made up of a single module.
\begin{itemize}
  \item \textbf{axis\_1553} Interface AXIS to PMOD1553 device.
\end{itemize}

\subsection{Encoding Method}
\par


\section{Building}

\par
The AXIS 1553 is written in Verilog 2001. It should synthesize in any modern FPGA software. The core comes as a fusesoc packaged core and can be included in any other core. Be sure to make sure you have meet the dependencies listed in the previous section. Linting is performed by verible using the lint target.

\subsection{fusesoc}
\par
Fusesoc is a system for building FPGA software without relying on the internal project management of the tool. Avoiding vendor lock in to Vivado or Quartus.
These cores, when included in a project, can be easily integrated and targets created based upon the end developer needs. The core by itself is not a part of
a system and should be integrated into a fusesoc based system. Simulations are setup to use fusesoc and are a part of its targets.

\subsection{Source Files}

\input{src/fusesoc/files_fusesoc_info.tex}

\subsection{Targets}

\input{src/fusesoc/targets_fusesoc_info.tex}de

\subsection{Directory Guide}

\par
Below highlights important folders from the root of the directory.

\begin{enumerate}
  \item \textbf{docs} Contains all documentation related to this project.
    \begin{itemize}
      \item \textbf{manual} Contains user manual and github page that are generated from the latex sources.
    \end{itemize}
  \item \textbf{src} Contains source files for the core
  \item \textbf{tb} Contains test bench files for cocotb
    \begin{itemize}
      \item \textbf{cocotb} testbench files
    \end{itemize}
\end{enumerate}

\newpage

\section{Simulation}
\par
There are a few different simulations that can be run for this core.

\subsection{cocotb}
\par
To use the cocotb tests you must install the following python libraries.
\begin{lstlisting}[language=bash]
  $ pip install cocotb
  $ pip install cocotbext-axi
  $ pip install cocotbext-mil_std_1553
\end{lstlisting}

Then you must use the cocotb sim target. The targets above can be run with various bus and fifo parameters.

\begin{lstlisting}[language=bash]
  $ fusesoc run --target sim_cocotb AFRL:device_converter:axis_1553:1.0.0
\end{lstlisting}


\newpage

\section{Module Documentation} \label{Module Documentation}

\begin{itemize}
\item \textbf{axis\_1553\_encoder} Interfaces AXIS to the PMOD1553.\\
\end{itemize}
The next sections document the module in great detail.

