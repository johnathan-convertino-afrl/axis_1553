\begin{titlepage}
  \begin{center}

  {\Huge AXIS\_1553}

  \vspace{25mm}

  \includegraphics[width=0.90\textwidth,height=\textheight,keepaspectratio]{img/AFRL.png}

  \vspace{25mm}

  \today

  \vspace{15mm}

  {\Large Jay Convertino}

  \end{center}
\end{titlepage}

\tableofcontents

\newpage

\section{Usage}

\subsection{Introduction}

\par
AXIS 1553 is a transmit and receive for the MIL-STD-1553 bus. This core can run at full duplex even though
MIL-STD-1553 is half duplex. This core provides a simple axis streaming interace that uses tuser to extend the
bus to allow for command and data syncs to be choosen, along with setting or indicating more than a 4us delay. 
There are additional signals for frame errors, sync only, and parity errors. These can be used to manage issues
that present themeselves to the core. Internally this core generates its own enables to cycle data out at its 
synthetic sample rate. Data is transmitted based on a 1 MHz clock, but is sythetically generated to 2 Mhz sample rate.

\subsection{Dependencies}

\par
The following are the dependencies of the cores.

\begin{itemize}
  \item fusesoc 2.X
  \item iverilog (simulation)
  \item cocotb (simulation)
\end{itemize}

\input{src/fusesoc/depend_fusesoc_info.tex}

\subsection{In a Project}
\par
Connect the device to your AXIS bus. TUSER is used to set or check various data status such as command/data packet mode.

\par
TDATA input contains the 16 bit data payload. TUSER is a 5 bit command register
that can take input or provide output that is a description what type of sync (command or data) 
and other options described below.

TUSER = {S,D,TYY} (4 downto 0)
\begin{itemize}
\item TYY = TYPE OF DATA
  \begin{itemize}
    \item 000 N/A (IF DATA VALID, SYNC IS NOT VALID AND DATA IS NOT AS WELL)
    \item 001 REG (NOT IMPLEMENTED)
    \item 010 DATA
    \item 100 CMD/STATUS
  \end{itemize}
  \item D = DELAY ENABLED
  \begin{itemize}
    \item 1 = 4 us delay enabled.
    \item 0 = no delay between transmissions.
  \end{itemize}
  \item S = SYNC ONLY
  \begin{itemize}
    \item 1 = Sync only detected
    \item 0 = Standard message
  \end{itemize}
\end{itemize}

\section{Architecture}
\par
This core is made up of the following modules.
\begin{itemize} 
  \item \textbf{axis\_1553} Interface AXIS to PMOD1553 device.
  \item \textbf{mod\_clock\_ena\_gen} Generate enable pulse at required sample rate.
  \item \textbf{PISO} Parallel input serial output.
  \item \textbf{SIPO} Serial input parallel output.
\end{itemize}

\subsection{MIL-STD-1553}
\par
In this core the data is encoded using Manchester II (G.E. Thomas) method. IEEE 802.3 uses a XOR with a clock and data. Manchester II uses a XNOR clock and data. The sync pulse is a non-machester compliant
part of the transmission that is only for detecting the start of the frame and what type of frame is incoming. The two types are data, and command/status. All messages are terminated with a odd parity bit.

\subsection{Encoding (transmit) Method}
\par
MIL-STD-1553 data is generated using a combinatorial process for the XNOR. This XNOR is performed using the data and a synthetic clock of 1 MHz at a 2 MHz sample rate (minimum for the digital waveform).
This is then contatenated with a pre-defined sync that is choosen based upon TUSER, with a generated parity bit XNOR with the synth clock. This is then loaded into the PISO core and cycled out at the sample
rate by the mod\_clock\_ena\_gen for tx enable pulse. If TUSER had set a delay the mod\_clock\_ena\_gen for tx will be put into a hold state till the counter has finished. Once all the data has been sent the
AXIS input will become ready again. The transmit output will set the differential lines to zero meaning there is no difference in output and there is no data. Transmit is activated using a active high output
for half duplex operation if needed.

\subsection{Decoding (receive) Method}
\par
MIL-STD-1553 data is input into the SIPO core. The mod\_clock\_ena\_gen for RX enable is cleared and kept on hold till the receive input is in a differential state. The state of the signals being identical means
there is no data being received (or transmitted). A few different conditions can arise. If the counter is less then the needed number of bits, and the diff in RX is no longer present then a sync only detection is
made or a frame error has happened. If the total number of bits are captured then the combinatorial XNOR decoder has its output sampled and placed into tdata and tuser properly. If the delay has been longer then 4us
since the last message the TUSER bit will be set to 1, otherwise there was no delay and it is 0.

\section{Building}

\par
The AXIS 1553 is written in Verilog 2001. It should synthesize in any modern FPGA software. The core comes as a fusesoc packaged core and can be included in any other core. Be sure to make sure you have meet the dependencies listed in the previous section. Linting is performed by verible using the lint target.

\subsection{fusesoc}
\par
Fusesoc is a system for building FPGA software without relying on the internal project management of the tool. Avoiding vendor lock in to Vivado or Quartus.
These cores, when included in a project, can be easily integrated and targets created based upon the end developer needs. The core by itself is not a part of
a system and should be integrated into a fusesoc based system. Simulations are setup to use fusesoc and are a part of its targets.

\subsection{Source Files}

\input{src/fusesoc/files_fusesoc_info.tex}

\subsection{Targets}

\input{src/fusesoc/targets_fusesoc_info.tex}de

\subsection{Directory Guide}

\par
Below highlights important folders from the root of the directory.

\begin{enumerate}
  \item \textbf{docs} Contains all documentation related to this project.
    \begin{itemize}
      \item \textbf{manual} Contains user manual and github page that are generated from the latex sources.
    \end{itemize}
  \item \textbf{src} Contains source files for the core
  \item \textbf{tb} Contains test bench files for cocotb
    \begin{itemize}
      \item \textbf{cocotb} testbench files
    \end{itemize}
\end{enumerate}

\newpage

\section{Simulation}
\par
There are a few different simulations that can be run for this core.

\subsection{cocotb}
\par
To use the cocotb tests you must install the following python libraries.
\begin{lstlisting}[language=bash]
  $ pip install cocotb
  $ pip install cocotbext-axi
  $ pip install cocotbext-mil_std_1553
\end{lstlisting}

Then you must use the cocotb sim target. The targets above can be run with various bus and fifo parameters.

\begin{lstlisting}[language=bash]
  $ fusesoc run --target sim_cocotb AFRL:device_converter:axis_1553:1.0.0
\end{lstlisting}


\newpage

\section{Module Documentation} \label{Module Documentation}

\begin{itemize}
\item \textbf{axis\_1553} Interfaces AXIS to the PMOD1553.\\
\end{itemize}
The next sections document the module in great detail.

